\documentclass{nagauth}
%\documentclass{nagauth}
\usepackage[utf8]{inputenc}
\usepackage[T1]{fontenc}
\usepackage[english]{babel}
\usepackage{floatrow}
\floatsetup[table]{style=plaintop,font=small}
%\captionsetup{font=small}

%\def\baselinestretch{2}

\usepackage{ifdraft}
\ifoptionfinal
{\newcommand{\draft}[1]{}}
{\newcommand{\draft}[1]{{#1}}}


% microtypography, makes stuff prettier
% expansion doesn't work with the current combo of fonts+diff
\usepackage[expansion=false]{microtype}

\usepackage[numbers]{natbib}
\usepackage{graphicx,subfig}
\usepackage{amsmath,amssymb}
\allowdisplaybreaks[1] % allow, but prefer not to, page break in a multiline eqn
\usepackage{tikz}
\usetikzlibrary{snakes,patterns}

\usepackage{booktabs}

%\usepackage[misc]{ifsym} \newcommand\envelope\Letter

\usepackage{color}
\definecolor{dimbg}{gray}{.93}
\definecolor{dimfg}{gray}{.2}
\makeatletter\newenvironment{graybox}{%
  \begin{lrbox}{\@tempboxa}\begin{minipage}{\columnwidth}}{\end{minipage}\end{lrbox}%
  \colorbox{dimbg}{\usebox{\@tempboxa}}
}\makeatother

\newcommand   {\R}[1]   {\text{#1}}       % (math) roman
\newcommand   {\del}    {\partial}
\newcommand   {\V}[1]   {\boldsymbol{#1}}            % vector
\newcommand   {\VV}[1]  {\mathrm{\bf #1}}           % tensor
\newcommand   {\D}      {\,\mathrm{d}}

\newcommand   {\X}[1]   {\mathrm{#1}} % vector (1D matrix)
\newcommand   {\M}[1]   {\mathrm{#1}} % matrix
\newcommand   {\tM}[1]  {\tilde{\M{#1}}}
\newcommand   {\MM}[1]  {\mathcal{#1}} % block matrix

\newcommand   {\T}[2][]   {#2^{\!\mathsf{#1T}}}
\newcommand   {\norm}[1]{\lVert#1\rVert}

\newcommand   {\prev}[1]{\hat{#1}}

\let\oLambda\Lambda	\renewcommand{\Lambda}{\VV{\oLambda}}
\let\onabla\nabla	\renewcommand{\nabla}{\V{\onabla}}

\newcommand{\U}{\V{u}}
\newcommand{\vD}{\V{v}_\R{D}}
\newcommand{\Pf}{p_\R{f}}
\newcommand{\Ps}{p_\R{s}}

\DeclareMathOperator{\diag}{diag}
\DeclareMathOperator{\trace}{Tr}

\newcommand {\refsec}[1] {\ERROR}
\newcommand {\refeq}[1]  {Equation~\eqref{eq:#1}}
\newcommand {\refeqs}[2] {Equations~\eqref{eq:#1} and~\eqref{eq:#2}}
\newcommand {\refeqr}[2] {Equations~\eqref{eq:#1}--\eqref{eq:#2}}
\newcommand {\Refeq}[1]  {Equation~\eqref{eq:#1}}
\newcommand {\Refeqr}[2] {Equations~\eqref{eq:#1}--\eqref{eq:#2}}
\newcommand {\Refeqs}[2] {Equations~\eqref{eq:#1} and~\eqref{eq:#2}}
\newcommand {\reffig}[1] {Figure~\ref{fig:#1}}
\newcommand {\reffigs}[2]{Figures~\ref{fig:#1} and~\ref{fig:#2}}
\newcommand {\Reffig}[1] {Figure~\ref{fig:#1}}
\newcommand {\reftab}[1] {Table~\ref{tab:#1}}
\newcommand {\Reftab}[1] {Table~\ref{tab:#1}}
\newcommand {\reftabs}[2]{Tables~\ref{tab:#1} and~\ref{tab:#2}}


\newcommand  {\todo}[1]    {\draft{\noindent\begin{graybox}{\small #1}\end{graybox}}}
\newcommand  {\todop}[1]   {\draft{\colorbox{dimbg}{#1}\hspace{-.5ex}}}
\newcommand  {\todos}[2][] {\todo{{#1}\begin{itemize}{#2}\end{itemize}}}
\newcommand  {\ignore}[1]  {}

\newcommand{\fig}[3][] {
  \begin{figure}
    \ffigbox[\FBwidth]
    {\includegraphics[#1]{data/#2}}
    {\caption{\draft{\bf\tiny[#2] }#3}\label{fig:#2}}
  \end{figure}
}

\newcommand{\twofig}[6][] {
  \begin{figure}
    \ffigbox[#1]
    {
      \subfloat[\draft{\bf\tiny[#2] }#3]{\label{fig:#2}\includegraphics[width=.45\textwidth]{data/#2}}
      \subfloat[\draft{\bf\tiny[#4] }#5]{\label{fig:#4}\includegraphics[width=.45\textwidth]{data/#4}}
    }
    {\caption{#6}\label{fig:#2:#4}}
  \end{figure}
}

\newcommand{\allfig}[3][] {
  \begin{figure}
    \ffigbox[#1]
    {
      \subfloat[$\epsilon=10^{-1}$]{\label{fig:#2:a}\includegraphics[width=.29\textwidth]{data/#2,EPS-1}}
      \subfloat[$\epsilon=10^{-4}$]{\label{fig:#2:b}\includegraphics[width=.29\textwidth]{data/#2,EPS-4}}
      \subfloat[$\epsilon=10^{-14}$]{\label{fig:#2:c}\includegraphics[width=.29\textwidth]{data/#2,EPS-14}}
    }
    {\caption{\draft{\bf\tiny[#2] }#3}\label{fig:#2}}
  \end{figure}}

\pgfdeclarepatternformonly%
{inside domain}
{ % bounding box lower left
  \pgfqpoint{-1pt}{-1pt}
}{% bounding box upper right
  \pgfqpoint{10pt}{10pt}
}{% step vector (cell spacing)
  \pgfqpoint{9pt}{9pt}
}{% pattern code
  \pgfsetlinewidth{0.4pt}
  \pgfpathmoveto{\pgfqpoint{0pt}{0pt}}
  \pgfpathlineto{\pgfqpoint{9.1pt}{9.1pt}}
  \pgfusepath{stroke}
}

\newlength{\fracheight}
\settoheight{\fracheight}{$\frac{T}{x}$}
\newcommand{\fracspacer}{\rule[-\fracheight]{0pt}{0pt}}

%\newcommand{\version}{\today{} $\cdot$ \input{version}}
\date{\today}

\newcommand{\ptikz}[1]{\protect\tikz{\protect #1}}

\newcommand{\BJ}{\ensuremath{\R{SJ}}}
\newcommand{\GS}{\ensuremath{\R{GS}}}
\newcommand{\SGS}{\ensuremath{\R{SGS}}}
\newcommand{\cBJ}{\ensuremath{\R{CJ}}}
\newcommand{\cSGS}{\ensuremath{\R{CSGS}}}
\newcommand{\sBJ}{\ensuremath{\R{GJ($-1$)}}}
\newcommand{\sGS}{\ensuremath{\R{GGS}}}
\newcommand{\sSGS}{\ensuremath{\R{GSGS}}}
\newcommand{\GJ}[1]{\ensuremath{\R{GJ($#1$)}}}
\newcommand{\sSGScg}{\ensuremath{\R{GSGS/CG}}}

\newsavebox{\mybox}
\newlength{\myboxlen}
\newcommand{\capbox}[3][-1ex]{
  \sbox{\mybox}{#3}
  \settowidth{\myboxlen}{\usebox{\mybox}}
  \addtolength{\myboxlen}{#1}
  \tabcap{\myboxlen}
  \ifthenelse{\equal{#2}{}}%
  {}%
  {\caption{#2}}
  \usebox{\mybox}
}
