%% -*- mode: LaTeX -*-
\documentclass{nagauth}
%\documentclass{nagauth}
\usepackage[utf8]{inputenc}
\usepackage[T1]{fontenc}
\usepackage[english]{babel}
\usepackage{floatrow}
\floatsetup[table]{style=plaintop,font=small}
%\captionsetup{font=small}

%\def\baselinestretch{2}

\usepackage{ifdraft}
\ifoptionfinal
{\newcommand{\draft}[1]{}}
{\newcommand{\draft}[1]{{#1}}}


% microtypography, makes stuff prettier
% expansion doesn't work with the current combo of fonts+diff
\usepackage[expansion=false]{microtype}

\usepackage[numbers]{natbib}
\usepackage{graphicx,subfig}
\usepackage{amsmath,amssymb}
\allowdisplaybreaks[1] % allow, but prefer not to, page break in a multiline eqn
\usepackage{tikz}
\usetikzlibrary{snakes,patterns}

\usepackage{booktabs}

%\usepackage[misc]{ifsym} \newcommand\envelope\Letter

\usepackage{color}
\definecolor{dimbg}{gray}{.93}
\definecolor{dimfg}{gray}{.2}
\makeatletter\newenvironment{graybox}{%
  \begin{lrbox}{\@tempboxa}\begin{minipage}{\columnwidth}}{\end{minipage}\end{lrbox}%
  \colorbox{dimbg}{\usebox{\@tempboxa}}
}\makeatother

\newcommand   {\R}[1]   {\text{#1}}       % (math) roman
\newcommand   {\del}    {\partial}
\newcommand   {\V}[1]   {\boldsymbol{#1}}            % vector
\newcommand   {\VV}[1]  {\mathrm{\bf #1}}           % tensor
\newcommand   {\D}      {\,\mathrm{d}}

\newcommand   {\X}[1]   {\mathrm{#1}} % vector (1D matrix)
\newcommand   {\M}[1]   {\mathrm{#1}} % matrix
\newcommand   {\tM}[1]  {\tilde{\M{#1}}}
\newcommand   {\MM}[1]  {\mathcal{#1}} % block matrix

\newcommand   {\T}[2][]   {#2^{\!\mathsf{#1T}}}
\newcommand   {\norm}[1]{\lVert#1\rVert}

\newcommand   {\prev}[1]{\hat{#1}}

\let\oLambda\Lambda	\renewcommand\Lambda{\VV{\oLambda}}
\let\onabla\nabla	\renewcommand\nabla{\V{\onabla}}

\newcommand\U{\V{u}}
\newcommand\vD{\V{v}_\R{D}}
\newcommand\Pf{p_\R{f}}
\newcommand\Ps{p_\R{s}}

\DeclareMathOperator{\diag}{diag}
\DeclareMathOperator{\trace}{Tr}

\newcommand {\refsec}[1] {\ERROR}
\newcommand {\refeq}[1]  {Equation~\eqref{eq:#1}}
\newcommand {\refeqs}[2] {Equations~\eqref{eq:#1} and~\eqref{eq:#2}}
\newcommand {\refeqr}[2] {Equations~\eqref{eq:#1}--\eqref{eq:#2}}
\newcommand {\Refeq}[1]  {Equation~\eqref{eq:#1}}
\newcommand {\Refeqr}[2] {Equations~\eqref{eq:#1}--\eqref{eq:#2}}
\newcommand {\Refeqs}[2] {Equations~\eqref{eq:#1} and~\eqref{eq:#2}}
\newcommand {\reffig}[1] {Figure~\ref{fig:#1}}
\newcommand {\reffigs}[2]{Figures~\ref{fig:#1} and~\ref{fig:#2}}
\newcommand {\Reffig}[1] {Figure~\ref{fig:#1}}
\newcommand {\reftab}[1] {Table~\ref{tab:#1}}
\newcommand {\Reftab}[1] {Table~\ref{tab:#1}}
\newcommand {\reftabs}[2]{Tables~\ref{tab:#1} and~\ref{tab:#2}}


\newcommand  {\todo}[1]    {\draft{\noindent\begin{graybox}{\small #1}\end{graybox}}}
\newcommand  {\todop}[1]   {\draft{\colorbox{dimbg}{#1}\hspace{-.5ex}}}
\newcommand  {\todos}[2][] {\todo{{#1}\begin{itemize}{#2}\end{itemize}}}
\newcommand  {\ignore}[1]  {}

\newcommand{\fig}[3][] {
  \begin{figure}
    \ffigbox[\FBwidth]
    {\includegraphics[#1]{data/#2}}
    {\caption{\draft{\bf\tiny[#2] }#3}\label{fig:#2}}
  \end{figure}
}

\newcommand{\twofig}[6][] {
  \begin{figure}
    \ffigbox[#1]
    {
      \subfloat[\draft{\bf\tiny[#2] }#3]{\label{fig:#2}\includegraphics[width=.45\textwidth]{data/#2}}
      \subfloat[\draft{\bf\tiny[#4] }#5]{\label{fig:#4}\includegraphics[width=.45\textwidth]{data/#4}}
    }
    {\caption{#6}\label{fig:#2:both}}
  \end{figure}
}

\pgfdeclarepatternformonly%
{inside domain}
{ % bounding box lower left
  \pgfqpoint{-1pt}{-1pt}
}{% bounding box upper right
  \pgfqpoint{10pt}{10pt}
}{% step vector (cell spacing)
  \pgfqpoint{9pt}{9pt}
}{% pattern code
  \pgfsetlinewidth{0.4pt}
  \pgfpathmoveto{\pgfqpoint{0pt}{0pt}}
  \pgfpathlineto{\pgfqpoint{9.1pt}{9.1pt}}
  \pgfusepath{stroke}
}

\newlength{\fracheight}
\settoheight{\fracheight}{$\frac{T}{x}$}
\newcommand{\fracspacer}{\rule[-\fracheight]{0pt}{0pt}}

\newcommand{\version}{\today{} $\cdot$ \input{version}}
\date{\today}

\newcommand{\ptikz}[1]{\protect\tikz{\protect #1}}

\newcommand{\BJ}{\ensuremath{\R{SJ}}}
\newcommand{\GS}{\ensuremath{\R{GS}}}
\newcommand{\SGS}{\ensuremath{\R{SGS}}}
\newcommand{\cBJ}{\ensuremath{\R{CJ}}}
\newcommand{\cSGS}{\ensuremath{\R{CSGS}}}
\newcommand{\sBJ}{\ensuremath{\R{GJ($-1$)}}}
\newcommand{\sGS}{\ensuremath{\R{GGS}}}
\newcommand{\sSGS}{\ensuremath{\R{GSGS}}}
\newcommand{\GJ}[1]{\ensuremath{\R{GJ($#1$)}}}
\newcommand{\sSGScg}{\ensuremath{\R{GSGS/CG}}}

\newsavebox{\mybox}
\newlength{\myboxlen}
\newcommand{\capbox}[3][-1ex]{
  \sbox{\mybox}{#3}
  \settowidth{\myboxlen}{\usebox{\mybox}}
  \addtolength{\myboxlen}{#1}
  \tabcap{\myboxlen}
  \ifthenelse{\equal{#2}{}}%
  {}%
  {\caption{#2}}
  \usebox{\mybox}
}


%@EXEC_ = r'%@ '
%@PREFIX_ = '\\'

%@ M  = r'\mathrm{%s}'
%@ R  = r'\text{%s}'
%@ X  = r'\M{%s}'
%@ MM = r'\mathcal{%s}'
%@ del = r'\partial'

\begin{document}

%\NAG{<first page>}{<last page>}{<volume>}{<issue>}{<yy>}
\NAG{0}{0}{0}{0}{00}
\runningheads{J.~B.~Haga \emph{et al.}}{Pressure oscillations in low-permeable porous media}
\noreceived{}%\received{<date>}
\norevised{}%\revised{<date>}
\noaccepted{}%\accepted{<date>}

\title{Pressure oscillations and locking in low-permeable and incompressible porous media}
\author{
  Joachim Berdal Haga\affil{1}\comma\affil{2}\comma\corrauth,
  Harald Osnes\affil{1}\comma\affil{2} and
  Hans Petter Langtangen\affil{1}\comma\affil{3}}
\address{%
\affilnum{1}\ Scientific Computing Department, Simula Research Laboratory, Norway\\
\affilnum{2}\ Department of Mathematics, University of Oslo, Norway\\
\affilnum{3}\ Department of Informatics, University of Oslo, Norway\\
\todop{\version}
}
\corraddr{%
J.~B.~Haga, Simula Research Laboratory, 
PO Box 134, N-1325 Lysaker, Norway. E-mail: jobh@simula.no}
%\footnotetext[2]{E-mail: jobh@simula.no}
\cgs{Statoil ASA}
\cgs{Norwegian Research Council}

\begin{abstract}
%
Large-scale simulations of flow in deformable porous media require
\end{abstract}
\keywords{Biot's consolidation; pressure oscillations; elasticity locking; low-permeable media; 
incompressible media; finite elements}

\section*{Introduction}
The locking phenomenon in elasticity is well known to cause spurious
oscillations in the (solid) pressure and erroneous displacement solutions for
nearly incompressible materials.
A similar phenomenon can be seen in poroelasticity when low-permeable regions
are present.
In the latter case, oscillations can be observed in the fluid pressure. In
[ref], Phillips et al.~argue that these phenomena come from the same
fundamental problem: the reduction of effective degrees of freedom in the
displacement solution when the finite elements are required to be (nearly)
divergence-free, $\nabla \cdot \U\approx 0$.

In the present paper, we test this hypothesis numerically by employing the
mixed finite element formulation for the displacement fielt in the poroelastic
equations.
Furthermore, we look at ways to overcome the pressure oscillations in
low-permeable materials without resorting to Discontinuous Galerkin methods
(which have been shown ito alleviate the problem [ibid.]).

\section{The mathematical model}

The equations describing poroelastic flow and deformation can be derived from
the principles of conservation of fluid mass and the balance of forces on the
porous matrix.
%
The linear poroelastic equations can, in the small-strains regime, be
expressed as
%
\begin{align}
\label{eq:sim1}
S\dot{p} - \nabla \cdot \Lambda \nabla p + \alpha \nabla \cdot \dot{\U} &= q, \\
\label{eq:sim2}
\nabla (\lambda+\mu) \nabla \cdot \U + \nabla \cdot \mu \nabla \U
 -  \alpha \nabla p &= \V{r}.
\end{align}
%
Here, we subsume body forces such as gravitational forces into the right-hand
side source terms $q$ and $\V{r}$.
The primary variables are $p$ for the fluid pressure and $\U$ for the
displacement of the porous medium, $S$ and $\Lambda$ are the fluid storage
coefficient and the flow mobility respectively, $\alpha$ is the Biot-Willis fluid/solid
coupling coefficient, and $\lambda$ and $\mu$ are the Lam\'e elastic parameters.

The fluid flow rate, or Darcy velocity, is often of particular interest in
poroelastic calculations.
It can be written as
\begin{equation}
  \vD= -\Lambda \nabla p,
\end{equation}
but due to the differential operator acting on $p$ it is of lower accuracy than
$p$ itself.
A natural extension is to introduce $\vD$ as a primary variable in a mixed
finite element formulation.
\Refeqr{sim1}{sim2} then expand to a coupled problem of two vector variables and one scalar
variable,
\begin{align}
\label{eq:simp1}
S\dot{p} + \nabla \cdot \vD + \alpha \nabla \cdot \dot{\U} &= q, \\
\label{eq:simp2}
  \Lambda^{-1} \vD + \nabla p &= 0,\\
\label{eq:simp3}
\nabla (\lambda+\mu) \nabla \cdot \U + \nabla \cdot \mu \nabla \U
 -  \alpha \nabla p &= \V{r}.
\end{align}
An alternative way to write \refeq{simp3} is
\begin{equation}
\nabla (\lambda+\mu) \nabla \cdot \U + \nabla \cdot \mu \nabla \U
 +  \alpha \Lambda^{-1}\vD = \V{r},
\end{equation}
which [must be tested, and removed?] probably leads to numerical instability
when $\Lambda\rightarrow 0$ and destroys the symmetry to boot.
We shall call this the mixed fluid pressure, or mixed-$p$, formulation.

In the field of (pure) elasticity, it is well understood that a
low-compressible material (with Poisson's ratio close to $0.5$) leads to
unphysical oscillations in the solid pressure field, and in some cases also to
a wrong solution for the calculated displacement.
This can be explained by $\lambda\rightarrow\infty$ in \refeq{sim2}, leading to
the requirement that $\nabla\cdot\U\rightarrow 0$ --- a requirement that is too
severe for common finite elements.
One way to overcome this obstacle is to introduce a new primary variable for
the solid pressure,
\begin{equation}
\label{eq:s}
s = -\lambda \nabla \cdot \U,
\end{equation}
and rewrite \refeq{sim2} as
\begin{align}
\nabla \mu \nabla \cdot \U + \nabla \cdot \mu \nabla \U
 - \nabla s - \alpha \nabla p &= \V{r},\\
\lambda^{-1}s+\nabla\cdot\U &= 0,
\end{align}
leading to a system of equations that is stable for low-compressible or even
incompressible materials.

In [ref], Phillips et al.~argue that the cause of pressure oscillations in
low-permeable media is the same: Consider \refeq{sim1}, discretised in time
with time step $\Delta t$ and with $S=q=0$.
Assume furthermore that we take one time step from a state with
$\nabla\cdot\U=0$.
Then,
\begin{equation}
  \nabla\cdot\U \approx \varepsilon \onabla^2 p,
\end{equation}
with $\varepsilon=\Delta t \Lambda / \alpha,$ which becomes very small for
short time steps and low permeabilities.
Again, we see the requirement for a divergence-free solution for the
displacement $\U$.

Assuming this the case, we can combine the mixed-$p$ and the mixed-$\U$
approaches, and introduce the solid pressure into the equation for the fluid
pressure, i.e., \refeqs{sim1}{simp1}.\footnote{As a practical issue, we note
  that this destroys the symmetry of the equations, with consequences for the
  choice of iterative solver for the discretised system.}
This doubly mixed system of two vector variables and two scalar variables can be
written as
\begin{align}
S\dot{p} + \nabla \cdot \vD + \alpha \lambda^{-1} \dot{s} &= q, \\
\label{eq:mixp}
  \Lambda^{-1} \vD + \nabla p &= 0,\\
\nabla \mu \nabla \cdot \U + \nabla \cdot \mu \nabla \U
 - \nabla s - \alpha \nabla p &= \V{r},\\
\label{eq:mixu}
\lambda^{-1}s+\nabla\cdot\U &= 0,
\end{align}
from which \refeq{mixp} and/or \refeq{mixu} may be eliminated, producing any of
the mixed formulations above.

\section{Losing the plot}

The domain of the first test problem is shown in \reffig{domain}.
Since we have not found any analytical solutions for the poroelastic equations
with different material parameters, we use this test problems in two ways: If
the load placed on the top boundary is uniform, it reduces to a one dimensional
problem, with a straightforward analytical solution.
Otherwise, we use the simple fact that the total equivalent volumetric stress,
\begin{equation}
\trace{\tilde{\sigma}} = s + \alpha p + 2\mu \nabla\cdot\U,
\end{equation}
should be continuous, smooth and monotonically non-increasing from top to bottom
under simple loading conditions.
This does not ensure that the numerical solution converges, but it is easy to
identify both oscillatory solutions and also some plausible but wrong ones
--- see, e.g., \reffig{asymm/Q0,RT1,Q1,-}b where each component field is smooth
but the fluid pressure is too high / solid pressure is too low.
For comparison, \reffig{asymm/Q0,RT1,Q1,Q0}b shows the correct (or at least
apparently so) solution: the load is balanced initially (in the top layer) by
solid pressure; then the fluid pressure takes over in the middle layer; and
finally, in the bottom layer, a combination of fluid/solid pressure and
volumetric strain.

\fig{domain}{The test domain. On the sides and bottom, no-flux conditions are
  imposed so that no fluid or solid movement is allowed in the normal
  direction. The top is drained with fluid pressure $p=0$ and an applied normal
  stress (load).}

\fig[width=\textwidth]{symm/Q1,-,Q1,-}{Constant load}
\fig[width=\textwidth]{symm/Q0,RT1,Q1,-}{Constant load, mixed-$p$}

\fig[width=\textwidth]{asymm/Q1,-,Q1,-}{Asymmetric load}
\fig[width=\textwidth]{asymm/Q0,RT1,Q1,-}{Asymmetric load, mixed-$p$}
\fig[width=\textwidth]{asymm/Q1,-,Q1,Q0}{Asymmetric load, mixed-$\U$}
\fig[width=\textwidth]{asymm/Q0,RT1,Q1,Q0}{Asymmetric load, mixed-both}


\section{Getting back on track}

Elastic locking only appears with rotation, compare \reffig{symm/Q1,-,Q1,-}
with \reffig{asymm/Q1,-,Q1,-}.

\fig[width=\textwidth]{symm/Q2,-,Q2,Q2}{}

Without rotation, but with explicit violation of inf-sup,
\reffig{symm/Q2,-,Q2,Q2}.

\fig[width=\textwidth]{symm/Q1,-,Q2,Q1-}{}
\fig[width=\textwidth]{symm/Q1-,Q2,Q2,Q1-}{}

\section*{Concluding remarks}



\acks


\bibliography{../references}
\bibliographystyle{wileyj}

\end{document}
