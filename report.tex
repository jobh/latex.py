%@OUTPUT_ = False
%@VERBOSE_ = 0
%@ABORT_ = False
%@PREFIX_ = '\\'

%@def fig(name, text, opt=None): print name
%@def twofig(n1, t1, n2, t2, text, opt=None): print n1+'\n'+n2
%@def allfig(name, text, opt=None):
%@    for i in [1,4,14]: print '%s,EPS-%d\n'%(name,i)
%@def includegraphics(name, opt=None): print name

\documentclass{nagauth}
\usepackage[utf8]{inputenc}
\usepackage[T1]{fontenc}
\usepackage[english]{babel}
\usepackage{floatrow}
\floatsetup[table]{style=plaintop,font=small}
%\captionsetup{font=small}

%\def\baselinestretch{2}

\usepackage{ifdraft}
\ifoptionfinal
{\newcommand{\draft}[1]{}}
{\newcommand{\draft}[1]{{#1}}}


% microtypography, makes stuff prettier
% expansion doesn't work with the current combo of fonts+diff
\usepackage[expansion=false]{microtype}

\usepackage[numbers]{natbib}
\usepackage{graphicx,subfig}
\usepackage{amsmath,amssymb}
\allowdisplaybreaks[1] % allow, but prefer not to, page break in a multiline eqn
\usepackage{tikz}
\usetikzlibrary{snakes,patterns}

\usepackage{booktabs}

%\usepackage[misc]{ifsym} \newcommand\envelope\Letter

\usepackage{color}
\definecolor{dimbg}{gray}{.93}
\definecolor{dimfg}{gray}{.2}
\makeatletter\newenvironment{graybox}{%
  \begin{lrbox}{\@tempboxa}\begin{minipage}{\columnwidth}}{\end{minipage}\end{lrbox}%
  \colorbox{dimbg}{\usebox{\@tempboxa}}
}\makeatother

%@expect_version(1.00)

\newcommand{\python}[1]{}
%@python = ''

%@def draft(x):
%@    cl = _latex.get('documentclass', [])
%@    if len(cl) > 1 and 'final' in cl[1]:
%@        return ''
%@    else:
%@        return '{%s}'%x

\newcommand   {\R}[1]   {\text{#1}}       % (math) roman
\newcommand   {\del}    {\partial}
\newcommand   {\V}[1]   {\boldsymbol{#1}}            % vector
\newcommand   {\VV}[1]  {\mathrm{\bf #1}}           % tensor
\newcommand   {\D}      {\,\mathrm{d}}

\newcommand   {\X}[1]   {\mathrm{#1}} % vector (1D matrix)
\newcommand   {\M}[1]   {\mathrm{#1}} % matrix
\newcommand   {\tM}[1]  {\tilde{\M{#1}}}
\newcommand   {\MM}[1]  {\mathcal{#1}} % block matrix

\newcommand   {\T}[2][] {#2^{\!\mathsf{#1T}}}
\newcommand   {\norm}[1]{\lVert#1\rVert}

\newcommand   {\prev}[1]{\hat{#1}}

%@Lambda = r'\VV{%s}' % _escape(r'\Lambda')
%@nabla  = r'\V{%s}' % _escape(r'\nabla')
%@onabla = _escape(r'\nabla')

\newcommand{\U} {\V{u}}
\newcommand{\vD}{\V{v}_\R{D}}
\newcommand{\Pf}{p_\R{f}}
\newcommand{\Ps}{p_\R{s}}

\DeclareMathOperator{\diag}{diag}
\DeclareMathOperator{\trace}{Tr}

\newcommand {\refeq}[1]  {Equation~\eqref{eq:#1}}
\newcommand {\refeqs}[2] {Equations~\eqref{eq:#1} and~\eqref{eq:#2}}
\newcommand {\refeqr}[2] {Equations~\eqref{eq:#1}--\eqref{eq:#2}}
\newcommand {\Refeq}[1]  {Equation~\eqref{eq:#1}}
\newcommand {\Refeqr}[2] {Equations~\eqref{eq:#1}--\eqref{eq:#2}}
\newcommand {\Refeqs}[2] {Equations~\eqref{eq:#1} and~\eqref{eq:#2}}
\newcommand {\reffig}[1] {Figure~\ref{fig:#1}}
\newcommand {\reffigs}[2]{Figures~\ref{fig:#1} and~\ref{fig:#2}}
\newcommand {\Reffig}[1] {Figure~\ref{fig:#1}}
\newcommand {\reftab}[1] {Table~\ref{tab:#1}}
\newcommand {\Reftab}[1] {Table~\ref{tab:#1}}
\newcommand {\reftabs}[2]{Tables~\ref{tab:#1} and~\ref{tab:#2}}

\newcommand  {\todo}[1]    {\draft{\noindent\begin{graybox}{\small #1}\end{graybox}}}
\newcommand  {\todop}[1]   {\draft{\colorbox{dimbg}{#1}\hspace{-.5ex}}}
\newcommand  {\todos}[2][] {\todo{{#1}\begin{itemize}{#2}\end{itemize}}}
\newcommand  {\ignore}[1]  {}

\python
{%@
def fig(name, caption, width=''):
    :\begin{figure}
    :\ffigbox[\FBwidth]
    :{\includegraphics[#(width)]{data/#(name)}}
    :{\caption{\draft{\bf\tiny[#(name)] }#(caption)}\label{fig:#(name)}}
    :\end{figure}
}%@

\python
{%@
def twofig(name1, cap1, name2, cap2, cap_both, cap_width=''):
    :\begin{figure}
    :\ffigbox[#(cap_width)]
    :{
    :\subfloat[\draft{\bf\tiny[#(name1)] }#(cap1)]%
    :{\label{fig:#(name1)}\includegraphics[width=.45\textwidth]{data/#(name1)}}
    :\subfloat[\draft{\bf\tiny[#(name2)] }#(cap2)]%
    :{\label{fig:#(name2)}\includegraphics[width=.45\textwidth]{data/#(name2)}}
    :}
    :{\caption{#(cap_both)}\label{fig:#(name1):#(name2)}}
    :\end{figure}
}%@

\python
{%@
def allfig(name, caption, cap_width=''):
    :\begin{figure}
    :\ffigbox[#(cap_width)]
    :{
    :\subfloat[$\epsilon=10^{-1}$]{\label{fig:#(name):a}\includegraphics[width=.29\textwidth]{data/#(name),EPS-1}}
    :\subfloat[$\epsilon=10^{-4}$]{\label{fig:#(name):b}\includegraphics[width=.29\textwidth]{data/#(name),EPS-4}}
    :\subfloat[$\epsilon=10^{-14}$]{\label{fig:#(name):c}\includegraphics[width=.29\textwidth]{data/#(name),EPS-14}}
    :}
    :{\caption{\draft{\bf\tiny[#(name)] }#(caption)}\label{fig:#(name)}}
    :\end{figure}
}%@

\pgfdeclarepatternformonly{inside domain}%
% bounding box lower left
{\pgfqpoint{-1pt}{-1pt}}%
% bounding box upper right
{\pgfqpoint{10pt}{10pt}}%
% step vector (cell spacing)
{\pgfqpoint{9pt}{9pt}}%
% pattern code
{
  \pgfsetlinewidth{0.4pt}
  \pgfpathmoveto{\pgfqpoint{0pt}{0pt}}
  \pgfpathlineto{\pgfqpoint{9.1pt}{9.1pt}}
  \pgfusepath{stroke}
}

\newlength{\fracheight}
\settoheight{\fracheight}{$\frac{T}{x}$}
\newcommand{\fracspacer}{\rule[-\fracheight]{0pt}{0pt}}

\python
{%@
def version():
    try:
        # For regression testing purposes.
        cmd = 'git describe --all --long %s' % _REV
    except:
        cmd = 'git describe --dirty=+ --all --long'
    ver = shell_eval(cmd).strip()
    if ver[-1] == '+':
        ver = ver[-8:]
    else:
        ver = ver[-7:]
    try:
        # If BASE_REV is set, output in latexdiff format
        base_ver = shell_eval('git rev-parse --short '+BASE_REV).strip()
        return : \today{} $\cdot$ \DIFdel{#(base_ver)} \DIFadd{#(ver)}
    except:
        return : \today{} $\cdot$ #(ver)
}%@

\date{\today}

\newcommand{\ptikz}[1]{\protect\tikz{\protect #1}}

\newcommand{\BJ}{\ensuremath{\R{SJ}}}
\newcommand{\GS}{\ensuremath{\R{GS}}}
\newcommand{\SGS}{\ensuremath{\R{SGS}}}
\newcommand{\cBJ}{\ensuremath{\R{CJ}}}
\newcommand{\cSGS}{\ensuremath{\R{CSGS}}}
\newcommand{\sBJ}{\ensuremath{\R{GJ($-1$)}}}
\newcommand{\sGS}{\ensuremath{\R{GGS}}}
\newcommand{\sSGS}{\ensuremath{\R{GSGS}}}
\newcommand{\GJ}[1]{\ensuremath{\R{GJ($#1$)}}}
\newcommand{\sSGScg}{\ensuremath{\R{GSGS/CG}}}

\newsavebox{\mybox}
\newlength{\myboxlen}
\python
{%@
def capbox(caption, width, adj='-1ex'):
    :\sbox{\mybox}{width}
    :\settowidth{\myboxlen}{\usebox{\mybox}}
    :\addtolength{\myboxlen}{#(adj)}
    :\tabcap{\myboxlen}
    if caption:
        :\caption{#(caption)}
    :\usebox{\mybox}
}%@



\begin{document}

%\NAG{<first page>}{<last page>}{<volume>}{<issue>}{<yy>}
\NAG{0}{0}{0}{0}{00}
\runningheads{J.~B.~Haga \emph{et al.}}{Pressure oscillations in low-permeable porous media}
\noreceived{}%\received{<date>}
\norevised{}%\revised{<date>}
\noaccepted{}%\accepted{<date>}

\title{Pressure oscillations and locking in low-permeable and incompressible porous media}
\author{
  Joachim Berdal Haga\affil{1}\comma\affil{2}\comma\corrauth,
  Harald Osnes\affil{1}\comma\affil{2} and
  Hans Petter Langtangen\affil{1}\comma\affil{3}}
\address{%
\affilnum{1}\ Scientific Computing Department, Simula Research Laboratory, Norway\\
\affilnum{2}\ Department of Mathematics, University of Oslo, Norway\\
\affilnum{3}\ Department of Informatics, University of Oslo, Norway\\
\todop{\version}
}
\corraddr{%
J.~B.~Haga, Simula Research Laboratory, 
PO Box 134, N-1325 Lysaker, Norway. E-mail: jobh@simula.no}
%\footnotetext[2]{E-mail: jobh@simula.no}
\cgs{Statoil ASA}
\cgs{Norwegian Research Council}

\begin{abstract}
%
Large-scale simulations of flow in deformable porous media require

We argue that pressure oscillations in low-permeable porous media should be
understood as a failure to satisfy the inf-sup condition on the finite element
spaces, rather than as a failure to represent a nearly divergence-free
displacement solution (locking).
This understanding makes it possible to use the established library of
inf-sup-stable elements as a guideline for choosing finite element spaces
for Biot's equations.
\end{abstract}
\keywords{Biot's consolidation; pressure oscillations; elastic locking; low-permeable media; 
incompressible media; finite elements}

\section{Introduction}
The locking phenomenon in elasticity is well known to cause spurious
oscillations in the (solid) pressure and erroneous displacement solutions for
nearly incompressible materials.
A similar phenomenon can be seen in poroelasticity when low-permeable regions
are present.
In the latter case, oscillations can be observed in the fluid pressure. In
[ref], Phillips et al.~argue that these phenomena come from the same
fundamental problem: the reduction of effective degrees of freedom in the
displacement solution when the finite elements are required to be (nearly)
divergence-free, $\nabla \cdot \U\approx 0$.

In the present paper, we test this hypothesis numerically by employing the
mixed finite element formulation for the displacement fielt in the poroelastic
equations.
Furthermore, we look at ways to overcome the pressure oscillations in
low-permeable materials without resorting to Discontinuous Galerkin methods
(which have been shown ito alleviate the problem [ibid.]).

\section{The mathematical model}

The equations describing poroelastic flow and deformation can be derived from
the principles of conservation of fluid mass and the balance of forces on the
porous matrix.
%
The linear poroelastic equations can, in the small-strains regime, be
expressed as
%
\begin{align}
\label{eq:sim1}
S\dot{\Pf} - \nabla \cdot \Lambda \nabla \Pf + \alpha \nabla \cdot \dot{\U} &= q, \\
\label{eq:sim2}
\nabla (\lambda+\mu) \nabla \cdot \U + \nabla \cdot \mu \nabla \U
 -  \alpha \nabla \Pf &= \V{r}.
\end{align}
%
Here, we subsume body forces such as gravitational forces into the right-hand
side source terms $q$ and $\V{r}$.
The primary variables are $\Pf$ for the fluid pressure and $\U$ for the
displacement of the porous medium, $S$ and $\Lambda$ are the fluid storage
coefficient and the flow mobility respectively, $\alpha$ is the Biot-Willis fluid/solid
coupling coefficient, and $\lambda$ and $\mu$ are the Lam\'e elastic parameters.

The fluid (Darcy) velocity is often of particular interest in poroelastic
calculations.
It can be written
\begin{equation}
  \label{eq:vD}
  \vD= -\Lambda \nabla \Pf,
\end{equation}
and represents the net macroscopic flux.
%
For the displacement equation, the main secondary quantity of interest is the
effective stress tensor,
\begin{align}
  \label{eq:sigma}
  \tilde{\VV{\sigma}} &= (\alpha \Pf + \Ps) \VV{I} +
  \mu\VV{(\nabla \U + \T{(\nabla \U)})},\\
  \intertext{which is written here using the solid pressure}
  \label{eq:Ps}
  \Ps &= -\lambda \nabla \cdot \U.
\end{align}

\subsubsection*{Weak discrete-in-time form.}

We employ a first-order backward finite difference method in time, which leads
to the discrete-time form of \refeq{sim1}
%
\begin{equation}
  \label{eq:time1}
  S \Pf - \Delta t\nabla \cdot \Lambda \nabla \Pf
  +  \nabla \cdot \U = q\Delta t + S \prev{\Pf} + \nabla \cdot \prev{\U}.
\end{equation}
%
where $(\prev{\Pf},\prev{\U})$ use values from the previous time step,
while unmarked variables are taken to be at the current time step.

Next, we rewrite \refeqs{sim2}{time1} in
weak form, using integration by parts to eliminate second derivatives.
The following relations must then be satisfied for all test functions $\pi$ and
$\V{\omega}$ in the domain $\Omega$:
%
\begin{gather}
  \label{eq:weakpress}
  \int_\Omega [
    \pi(
  S \Pf
  + \nabla \cdot \U)
  + \Delta t \nabla\pi \cdot \Lambda \nabla \Pf
  ] \D\Omega 
  = \int_\Omega \pi[
    q \Delta t
  + S \prev{\Pf}
  + \nabla \cdot \prev{\U}] \D\Omega
-\int_\Gamma \pi f_{\V{n}}\Delta t \D\Gamma,\\
%
\label{eq:weakdispl}
  \int_\Omega [
  (\nabla\cdot\V{\omega})((\lambda+\mu)(\nabla\cdot\U)
  -\alpha \Pf)
  +\nabla \V{\omega} : \mu  \nabla \U
  ] \D\Omega =
 - \int_\Omega \V{\omega} \cdot \V{r} \D\Omega + \int_\Gamma \V{\omega}
  \cdot \V{t}_{\V{n}} \D\Gamma.
\end{gather}
%
The normal flux $f_{\V{n}}=\vD\cdot\V n$ and normal stress $\V{t}_{\V{n}}$ on the boundary
$\Gamma$ appear here as natural boundary conditions. 
The discrete finite element approximation follows from solving
\refeqs{weakpress}{weakdispl} in finite-dimensional spaces.

\subsection{Mixed fluid pressure-velocity formulation}

In many applications of the poroelastic equations, we are mainly interested in
the flow of the fluid through the medium.
However, due to the differential operator acting on $\Pf$ it is of lower accuracy than
the pressure $\Pf$ itself.
Furthermore, the derivative is not continuous between elements, and hence the
fluid mass is not conserved.
A natural extension is then to introduce $\vD$ as an extra primary variable in a mixed
finite element formulation.
The order of accuracy is higher, and continuous elements for $\vD$ ensure mass
conservation.

By inserting \refeq{vD} into \refeq{sim1}, we get a coupled system of three
equations (of which two are vector equations). \Refeq{sim2} for solid
displacement is unchanged, and the equations for fluid flux and pressure are
\begin{align}
\label{eq:simp1}
S\dot{\Pf} + \nabla \cdot \vD + \alpha \nabla \cdot \dot{\U} &= q, \\
\label{eq:simp2}
  \Lambda^{-1} \vD + \nabla \Pf &= 0.
\end{align}
We shall call this the mixed fluid pressure, or mixed-$\Pf$, formulation.

\subsubsection*{Weak discrete-in-time form.}

Following the same conventions as in the previous section, \refeq{weakpress}
becomes the two coupled equations
\begin{gather}
  \label{eq:weakpressmix}
  \int_\Omega \pi[
  S \Pf +
  \Delta t \nabla \cdot \vD
  + \nabla \cdot \U
  ] \D\Omega 
  = \int_\Omega \pi[
  q \Delta t
  + S \prev{\Pf}
  + \nabla \cdot \prev{\U}] \D\Omega,\\
%
\label{eq:weakdarcy}
  \int_\Omega [
  \V{\phi} \cdot \Lambda^{-1}\vD - (\nabla \cdot \V{\phi})\Pf
  ]  \D\Omega
  + \int_\Gamma (\V{\phi}\cdot \V{n}) \Pf = 0,
\end{gather}
for any test functions $\pi$ and $\V{\phi}$ in the appropriate spaces.
The integration by parts in \refeq{weakdarcy} is undertaken to avoid
taking the derivative of $\Pf$, which is usually of low order (and possibly
piecewise constant).

\subsection{Mixed solid displacement-pressure formulation}

In the field of (pure) elasticity, it is well understood that a
low-compressible material (with Poisson's ratio close to $0.5$) leads to
unphysical oscillations in the solid pressure field, and in some cases also to
a wrong solution for the calculated displacement.
This can be explained by $\lambda$ becoming very large in \refeq{sim2}, leading to
the requirement that $\nabla\cdot\U\rightarrow 0$.
When this requirement is applied to standard finite elements, several degrees
of freedom become ``locked'', leaving too few degrees of freedom to represent
the correct solution.

One way to overcome this obstacle is to introduce a new primary variable for
the solid pressure, which is defined in \refeq{Ps}.
\Refeq{sim2} can then be rewritten as two coupled equations,
\begin{align}
\label{eq:mixu1}
\nabla \mu \nabla \cdot \U + \nabla \cdot \mu \nabla \U
 - \nabla \Ps - \alpha \nabla \Pf &= \V{r},\\
\label{eq:mixu2}
\lambda^{-1}\Ps+\nabla\cdot\U &= 0,
\end{align}
and combined with either \refeq{sim1} or with \refeqs{simp1}{simp2}.
In either case, we replace $\alpha \nabla \cdot \dot{\U}$ by $- \alpha
\lambda^{-1} \dot{\Ps}$; the reason for this change will be elucidated in later
sections.
The mixed solid displacement, or mixed-$\U$, formulation is stable for
low-compressible or even incompressible materials.

\subsubsection*{Weak discrete-in-time form.}

The weak form of \refeqr{mixu1}{mixu2} is
\begin{gather}
  \label{eq:weakpressmix2}
  \int_\Omega [\pi(
  S \Pf
  - \lambda^{-1} \Ps)
  + \Delta t \nabla\pi \cdot \Lambda \nabla \Pf
  ] \D\Omega 
  = \int_\Omega \pi[
   q \Delta t
  + S \prev{\Pf}
  - \lambda^{-1} \prev{\Ps}] \D\Omega
-\int_\Gamma \pi f_{\V{n}}\Delta t \D\Gamma,\\
%
\label{eq:weakdisplmix}
  \int_\Omega [
  (\nabla\cdot\V{\omega})(\mu(\nabla\cdot\U)
  -\alpha \Pf - \Ps)
  +\nabla \V{\omega} : \mu  \nabla \U
  ]  \D\Omega = - \int_\Omega \V{\omega} \cdot \V{r} \D\Omega + \int_\Gamma \V{\omega}
  \cdot \V{t}_{\V{n}} \D\Gamma,\\
%
\label{eq:weaks}
  \int_\Omega \psi[
     \lambda^{-1}\Ps + \nabla \cdot \U 
  ]  \D\Omega
  = 0,
\end{gather}
for test functions $\pi$, $\V{\omega}$ and $\psi$ in the appropriate spaces.
Finally, the trailing term on both sides of \refeq{weakpressmix2} can be replaced in the
same way as in \refeqr{weakpressmix}{weakdarcy} to produce a doubly mixed
system of two vector unknowns and two scalar unknowns.

\section{On the cause of pressure oscillations}

\subsection{Is it caused by locking?}

It is well known that spurious pressure oscillations may occur in low-permeable
regions in finite element calculations of the poroelastic equations [ref].
Phillips et al.~[ref] argue that the cause of these pressure oscillations 
is caused by the phenomenon known as \emph{locking} in pure elasticity.
Elastic locking appears when finite elements are asked to reproduce a
displacement field that is nearly divergence free, as
$\lambda\rightarrow\infty$ in \refeq{sim2} leads to $\nabla\cdot\U\rightarrow 0$.
Satisfying this with standard finite elements locks out many of the degrees of
freedom, in some cases to the extent that only (nearly) constant displacement
fields can be represented.
More commonly, the error in displacement is seen as nonphysical oscillations in
the stress field (i.e., the components of $\nabla\U$).

The argument by Phillips et al.~[] is that the same happens in poroelasticity,
in some conditions.
Consider \refeq{sim1} with constant permeability, discretised in time with time
step $\Delta t$ and with $S=q=0$.
Assume furthermore that we take one time step from a divergence-free initial
state, which is quite normal at the start of a simulation (when $\U=0$).
Then, \refeq{sim1} reduces to
\begin{equation}
  \nabla\cdot\U \approx \epsilon \onabla^2 \Pf,
\end{equation}
with $\epsilon=\Delta t \norm{\Lambda} / \alpha$.
The right-hand side becomes very small for short time steps and low permeabilities.
Again, we see the appearance of the requirement for a nearly divergence-free
solution for the displacement $\U$.

There are, however, some differences.
We illustrate these through an example where both kinds of problem are present: A
low-permeable region ($\Lambda\rightarrow 0$) and a low-compressible region
($\lambda\rightarrow\infty$).
This example is shown in \reffig{domain}.
The top layer is low-compressible, with $\lambda=\epsilon^{-1}$ for some $\epsilon<1$; the middle
layer is low-permeable, with $\Lambda=\epsilon\VV{I}$; while the bottom layer
has unit permeability and compressibility.
The boundary conditions at the sides and bottom are
\begin{align*}
  \U\cdot\V{n}&=0,\\
  \V{f}_{\V{n}}&=0,\\
\intertext{while the top boundary has boundary conditions}
\Pf&=0,\\
\V{t}_{\V{n}}&=F.
\end{align*}

\fig{domain}{The test domain. On the sides and bottom, no-flux conditions are
  imposed so that no fluid or solid movement is allowed in the normal
  direction. The top is drained with fluid pressure $\Pf=0$ and an applied normal
  stress (load).}

We use this test problems with two different choices of load $F$: Constant load
across the whole boundary, and load on just the right half.
To evaluate the numerical solutions, we use the fact that the total volumetric
effective stress,
\begin{equation}
\frac{\trace{\tilde{\sigma}}}{3} = \Ps + \alpha \Pf + 2\mu \nabla\cdot\U,
\end{equation}
should be continuous, smooth and monotonically non-increasing from top to bottom
under simple loading conditions. The correct solutions are shown in
\reffig{symm/Q0,RT1,Q1,-} for constant load and \reffig{asymm/Q0,RT1,CR1,Q0} for
non-constant load.
This does \emph{not} ensure that the numerical solution converges, but it is easy to
identify oscillatory solutions.
For small $\epsilon$, the load is balanced initially (in the top layer) by
solid pressure; then the fluid pressure takes over in the middle layer; and
finally, in the bottom layer, a combination of fluid/solid pressure and
volumetric strain takes the load.

\allfig{symm/Q0,RT1,Q1,-}{Correct solution for the constant-load problem.}
\allfig{asymm/Q0,RT1,CR1,Q0}{Correct solution for the variable-load problem.}

[The following is an outline of the argument / evidence. To be filled in with prose.]
\begin{itemize}
\item Elastic locking appears when there is rotation [ref for this?]. Pressure oscillations
  occurs even in the absence of rotation. See
  \reffig{symm/Q1,-,Q1,-,EPS-14:asymm/Q1,-,Q1,-,EPS-14}.
\item If elastic locking is the cause of pressure oscillations, it should be
  cured by the mixed-$\U$ formulation. But mixed-$\U$ only removes the
  oscillations in the top (incompressible) layer, not in the middle
  (impermeable) layer. See \reffig{asymm/P2,-,P2,-,EPS-14:asymm/P2,-,P2+,P1-,EPS-14}. This is
  probably the most powerful argument.
\item The impermeable layer does \emph{not} force $\U$ towards a constant. The
  incompressible layer does. See \reffig{u-locking}. Hence, locking is present
  in the incompressible but not in the impermeable layer.
\end{itemize}

\twofig{symm/Q1,-,Q1,-,EPS-14}{Constant load}{asymm/Q1,-,Q1,-,EPS-14}{Variable
  load}{Pressure oscillations even with constant load. $\epsilon=10^{-14}$}
\twofig{asymm/P2,-,P2,-,EPS-14}{Not
  mixed}{asymm/P2,-,P2+,P1-,EPS-14}{Mixed-$\U$}{Mixed displacement only
  effective for incompressible region. $\epsilon=10^{-14}$}
\fig[width=0.5\textwidth]{u-locking}{Locking in $\U_z$ (red) vs.~non-locking
  (green). Notice the nearly constant displacement in the top layer. $\epsilon=10^{-14}$}

The evidence points to pressure oscillations not being caused by locking due to
divergence-free $\U$.

\section{Is it caused by incompatible finite element spaces (in the inf-sup sense)?}

We introduce here the hypothesis that the pressure oscillations are caused by a
failure of the finite element spaces in the coupled system to fulfill the
Babuska-Brezzi [ref] inf-sup condition.
This is inspired by the similar roles that $\Pf$ and $\Ps$ play in relation to $\U$
in \refeqs{mixu1}{mixu2} and \eqref{eq:sim1}.
It is well known that only combinations of element spaces that satisfy the
inf-sup condition are stable for the $\U$-$\Ps$ system.
We shall present experimental evidence showing that the fluid pressure
oscillations are similar to the solid pressure oscillations that occur when the
inf-sup condition is violated.
Furthermore, we will show that using combinations of elements that are known to
be stable for Poisson and Stokes type problems, the oscillations disappear.
This continues in the following section, where we show the
convergence of some of these elements.

[Again, I just the outline of the arguments.]

\begin{itemize}
\item Intentional violation of the inf-sup condition for $\U$-$\Ps$ shows similar oscillations in the
  low-compressible region. [Argument from superficially similar symptoms is a
  weak one, I know, but still...]. This happens even in
  the absence of rotation, where ``normal'' displacement locking does not
  occur. See \reffig{symm/Q2,-,Q2,-,EPS-14:symm/Q2,-,Q2,Q2,EPS-14}.
\item Lowering the order of $\Pf$ and $\Ps$ removes oscillations in both
  layers. This is consistent with the current hypothesis, but not with the locking
  hypothesis, since we do not introduce more degrees of freedom in $\U$. See
  \reffig{symm/Q1,-,Q2,Q1-,EPS-14:symm/Q1-,Q2,Q2,Q1-,EPS-14}.
\item The present hypothesis (if correct) is good news, since the Discontinuous
  Galerkin method used by Phillips et al.~is computationally expensive. We can use first order
  Rannacher-Turek or Crouzeix-Raviart for displacement in combination with
  Raviart-Thomas for fluid velocity and piecewise constant pressure, or linear
  pressure with enriched (bubble) linear elements for displacement.
\end{itemize}

\twofig{symm/Q2,-,Q2,-,EPS-14}{Non-mixed}{symm/Q2,-,Q2,Q2,EPS-14}{Mixed-$\U$, with
  intentional inf-sup violation for $\U$-$\Ps$}{}
\twofig{symm/Q1,-,Q2,Q1-,EPS-14}{Mixed-$\U$}{symm/Q1-,Q2,Q2,Q1-,EPS-14}{Mixed-both}{Apparently
  correct solutions. When we introduce the mixed formulation also for the fluid
pressure/velocity, we can use discontinuous elements for $\Pf$. These can better
approximate the very steep gradients at the layer interfaces, and thus avoid
the spikes there.}


\section{Convergence testing}

[In this section, we study the convergence of some selected combinations of
  element spaces on the Barry-Mercer problem. Need to make a list of relevant
  elements.]

The Barry-Mercer problem, analysed in [ref], is illustrated in [todo figure ?].
It consists of a pulsating pressure point source within a square
two-dimensional domain of constant material parameters.
The boundary conditions are chosen to allow an analytical solution:
\begin{align*}
  \Pf &= 0, \\
  \U \times \V{n} &= 0\\
\intertext{at the sides, and the point source located at $\V{x}_0$ is given as}
q(\V{x},t) &= p_0 \delta(\V{x}-\V{x}_0) \sin t.
\end{align*}

It has been demonstrated [ref] that the Barry-Mercer problem exhibits pressure
oscillations for early times when $\epsilon=\norm{\Lambda} \Delta t \ll 1$.
To see this, we look at the first time step on the unit square, with $\Delta
t=10^{-2}$, $\Lambda=10^{-14}\VV{I}$, $p_0=1$ and $\V{x}_0=(0.25,0.25)$.
This combination of parameters induces severe oscillations in the numerical
solution when the mixed-$\Pf$ formulation with first order Raviart-Thomas
elements is used for the fluid and bilinear elements are used for the displacement.
\Reffig{barry-mercer/Q0,RT1,Q1,-,EPS-14:barry-mercer/Q0,RT1,CR1,-,EPS-14} shows
how a minor change, switching the $\U$ finite element from conforming bilinear
to nonconforming bilinear (Rannacher-Turek elements) removes the pressure
oscillations.

\twofig{barry-mercer/Q0,RT1,Q1,-,EPS-14}{a}{barry-mercer/Q0,RT1,CR1,-,EPS-14}{b}{}

\section{Concluding remarks}



\acks


\bibliography{../references}
\bibliographystyle{wileyj}

\end{document}


%% Local Variables:
%% mode: LaTeX
%% End:

