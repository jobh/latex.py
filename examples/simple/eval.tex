\begin{document}

% Eval is by default not visible as a macro due to the underline in
% the name (underline is considered as a separator). Make it available
% by binding to the macro name 'py':
%@py = _eval

$1+3/5 = @py{1+3/5}$

% We can even use the empty string as macro name. This is still fairly safe,
% in the sense that (a) the construction '@{...}' is not common in latex, and
% (b) it is even more uncommon that the '...' part is a valid python expression.
%@ = _eval

$2+4/6 = @{2+4/6}$

% The following is escaped, and hence not eval'ed.
%@escape = _escape
@escape{%
$3+5/7 = @{3+5/7}$
}%

% Another way is to specify that expansion only happens if there is exactly
% one argument (that is, ignore wrong argument count instead of raising an
% exception).
%@ = lambda *args: len(args)==1 and _eval(args[0]) or _ignore()
$2+4/6 = @{2+4/6}$
$2+4/6 = @{2+4/6}{}$


% While '=' is special-cased to allow using reserved python words as macro
% names, or even the empty string, these cannot be manipulated in other ways
% in standard python. How to --- for example --- delete the macro named ''?
% Two ways: Assign the '_ignore' function, or delete it from parser_scope
% directly. That is, either of the following two directives is sufficient:
%@ = _ignore
%@del get_scope()['']

% Note, however, that KeyErrors are normally reported so that missing macro
% definitions are identified. We can turn this off for now.
%@_args.verbose = 1

$4+6/8 = @{4+6/8}$

\end{document}
