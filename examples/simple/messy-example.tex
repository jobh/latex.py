The rules:

-- Lines starting with %@ are executed normally in python

-- Same for lines between "{%@" and "}%@" (note that any text on the lines
-- containing [{}]%@ will be ignored):
--  {%@
--  <python code>
--  }%@

-- execution happens after a block is closed (by a line without %@, or with }%@)

-- The stuff above is used to define MACROS as python functions,
-- these are replaced by the RETURN VALUE of the function upon use.

-- IMPORTANT: Macro names consists of alphanumerics plus '*' (no underscores).
-- Change _args.pattern if you must.

--- Typical definitions are like the following. These work exactly
--- the same, frac2 is only useful for simple substitutions though.
%@frac = lambda x,y: r'\mathrm{%s/%s}'%(x,y)

%@frac2 = r'\mathrm{%s/%s}'

%@def frac3(x,y):
%@    return r'\mathrm{%s/%s}'%(x,y)

%@def frac4(x,y):  # see below for explanation
%@    : \mathrm{#(x)/#(y)}

--- Macro arguments over multiple lines are ok IFF either the line breaks are
--- within the enclosing brackets, or line continuations (%) are used. Examples:

@frac{1}{2}

@frac  {1}  {2}

@frac{
1}%
{2}

--- As a special case, exactly one latex-style optional (square bracket) argument
--- can be used. It is moved to the end:
%@def testme(exp, base='10'):
%@    return '$%s^{%s}$' % (base, exp)
--- can be called as
@testme[10]{5.4}
@testme{5.4}
or
@testme{5.4}{10}

--- Some syntactic sugar for returning text:
%@def testme(exp, base='10'):
%@    : $#(base)^{#(exp)}$
--- (or equivalently)
%@def testme(exp, base='10'):
%@    return prepare_format(r'$#(base)^{#(exp)}$') % {'exp':exp, 'base':base}

--- This is also allowable to return text from blocks of code, and to define
--- variables.
%@name =: \emph{Your name here}
@name

%@for x in range(3):
%@    val = str(x)
%@    : \emph{#(val)}

--- But this one allows multiple lines of text. Let's try a more complicated
--- example. The special syntax ': <text>' is converted to
--- '_(r"<text>", local_args=locals())', which replaces '#(x)' with '%(x)s'
--- in <text>, and outputs <text>%locals.
{%@
def autotable(f,rows,caption=None):
    rows = [l.strip()+r'\\' for l in rows.split(r'\\')]
    output(r'\begin{matrix}{%s%s}' % (f[0], f[1]*rows[0].count('&')))
    :\toprule
    output(rows[0])
    :\midrule
    for row in rows[1:]:
        output(row)
    :\bottomrule
    :\end{matrix}
    if caption:
        :\caption{#(caption)}
}%@

--- This input:
(at)autotable{rc}{
x & y & z\\
r & b & x\\
a & f & f
}
--- expands into:
@autotable{rc}{
x & y & z\\
r & b & x\\
a & f & f
}

--- Working around python reserved words: These are allowed in *plain assignments* only.
--- like this:
%@def = r'Definition:'
%@del = r'\nabla'

--- The same technique can be used to create macros that contain characters that aren't
--- legal in identifiers in python ('*').
--- If you want to use this for functions, use lambda or aliases or
--- whatever. The following definitions are all equivalent.

%@def* = lambda x: "Definition: "+x
@def*{this}

%@def _def(x):
%@    : Definition: #(x)
%@def* = _def
@def*{this}

%@def* = r'Definition: %s'
@def*{this}

--- Other uses of reserved words must use the scope dict explicitly:
%@del get_scope()['def*']
