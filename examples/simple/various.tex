% This file can be parsed by both standard latex and by latex.py. It's just
% here to document a few tricks, which may or may not be useful.
% Note that:
% - the latex.py definition must follow the LaTeX definition
% - the difference in \pythononly, where LaTeX parses the argument on the
%   following line, but latex.py does not.

\documentclass{article}

\newcommand{\pythononly}[1]{} % make latex ignore python blocks
\newcommand{\latexonly}[1]{#1}
%@pythononly = ''             # make latex.py remove the 'python' keyword
%@latexonly = lambda x: ''

\begin{document}

% To see, run 'latex.py -L -o various.pdf various.tex; evince various.pdf'
\pythononly
{%@
: This document is prepared with \empy{latex.py}.
}%@
% The following doesn't require the \pythononly.
%@: Right?

% To see, run 'latex various.tex; evince various.pdf'
\latexonly%
{
This document is prepared with plain \LaTeX.
}

% Of course, we could simply have done the following.
\newcommand{\textprocessor}{plain \LaTeX}
%@textprocessor: \emph{latex.py}
To repeat: we used \textprocessor!

\end{document}
