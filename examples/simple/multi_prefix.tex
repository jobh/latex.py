\documentclass{article}

% Use another prefix for certain macros.

{%@
nabla = escape(r'\vec{\nabla}}')
with scope('@'):
    u = ensure_math(r'\vec{u}')
    Pf = ensure_math(r'P_\text{f}')
    S = ensure_math('S')
}%@

\begin{document}

We can use this secondary prefix for, e.g, marking unknowns in equations to
ensure they always use the same notation (even in plain text). We could have done this with
the standard prefix as well, but see below...

\begin{equation}
@S\dot{@Pf} \alpha \nabla \cdot \dot{@u} = @q
\end{equation}

But then we have to define them all, even the plain ones (@S and @q). Maybe this will work:

% Make any @X for undefined X return just 'X'.

{%@
with scope('@'):
    def __missing__(key, *args):
       if not key or args:
           ignore()
       return ensure_math(key)()
    del S  # Don't need it anymore, it's covered by __missing__
}%@

Now the equation is
\begin{equation}
@S\dot{@Pf} \alpha \nabla \cdot \dot{@u} = @q,
\end{equation}
with @q as a source term.

Yes! But there's a problem:

jobh@simula.no, jobh@{}simula.no, jobh@{simula.no}

We can work around this by protecting the part after @ with brackets, like above.
This can even be invisible in the output, like so:

% Make a macro '' (empty string) that just returns its escaped (single) argument
%@get_scope('@')[''] = lambda *args: len(args)==1 and escape('@')+args[0] or ignore()

Try again:
We can work around this by protecting the part after @ with brackets, like above.

jobh@simula.no, jobh@{}simula.no, jobh@{simula.no}

On my keyboard, there are several seldom-used symbols available, including '¬'
and '£', but these would be a problem for collaborators with different keyboard
layouts. Maybe '#' is an option, if we are careful to avoid code listings?

{%@
with scope('#'):
    @ensure_math
    def __missing__(key, *args):
        if (not key) or args or (current_match()[0] != '#'):
            ignore()
        for env in _latex['environment']:
           if env in ['lstlisting', 'verbatim']:
               ignore()
        usage_count[key] += 1
        return key
    R = ensure_math(r'\mathcal{R}')
}%@

\begin{equation}
(#x+#y)^2 = #x^2+#y^2+2#x#y,
\end{equation}
for any #x and #y in #R.

\end{document}
