%@expect_version(1.00)

\newcommand{\python}[1]{}
%@python = ''

%@def draft(x):
%@    cl = _latex.get('documentclass', [])
%@    if len(cl) > 1 and 'final' in cl[1]:
%@        return ''
%@    else:
%@        return '{%s}'%x

\newcommand   {\R}[1]   {\text{#1}}       % (math) roman
\newcommand   {\del}    {\partial}
\newcommand   {\V}[1]   {\boldsymbol{#1}}            % vector
\newcommand   {\VV}[1]  {\mathrm{\bf #1}}           % tensor
\newcommand   {\D}      {\,\mathrm{d}}

\newcommand   {\X}[1]   {\mathrm{#1}} % vector (1D matrix)
\newcommand   {\M}[1]   {\mathrm{#1}} % matrix
\newcommand   {\tM}[1]  {\tilde{\M{#1}}}
\newcommand   {\MM}[1]  {\mathcal{#1}} % block matrix

\newcommand   {\T}[2][] {#2^{\!\mathsf{#1T}}}
\newcommand   {\norm}[1]{\lVert#1\rVert}

\newcommand   {\prev}[1]{\hat{#1}}

%@Lambda = r'\VV{%s}' % _escape(r'\Lambda')
%@nabla  = r'\V{%s}' % _escape(r'\nabla')
%@onabla = _escape(r'\nabla')

\newcommand{\U} {\V{u}}
\newcommand{\vD}{\V{v}_\R{D}}
\newcommand{\Pf}{p_\R{f}}
\newcommand{\Ps}{p_\R{s}}

\DeclareMathOperator{\diag}{diag}
\DeclareMathOperator{\trace}{Tr}

\newcommand {\refeq}[1]  {Equation~\eqref{eq:#1}}
\newcommand {\refeqs}[2] {Equations~\eqref{eq:#1} and~\eqref{eq:#2}}
\newcommand {\refeqr}[2] {Equations~\eqref{eq:#1}--\eqref{eq:#2}}
\newcommand {\Refeq}[1]  {Equation~\eqref{eq:#1}}
\newcommand {\Refeqr}[2] {Equations~\eqref{eq:#1}--\eqref{eq:#2}}
\newcommand {\Refeqs}[2] {Equations~\eqref{eq:#1} and~\eqref{eq:#2}}
\newcommand {\reffig}[1] {Figure~\ref{fig:#1}}
\newcommand {\reffigs}[2]{Figures~\ref{fig:#1} and~\ref{fig:#2}}
\newcommand {\Reffig}[1] {Figure~\ref{fig:#1}}
\newcommand {\reftab}[1] {Table~\ref{tab:#1}}
\newcommand {\Reftab}[1] {Table~\ref{tab:#1}}
\newcommand {\reftabs}[2]{Tables~\ref{tab:#1} and~\ref{tab:#2}}

\newcommand  {\todo}[1]    {\draft{\noindent\begin{graybox}{\small #1}\end{graybox}}}
\newcommand  {\todop}[1]   {\draft{\colorbox{dimbg}{#1}\hspace{-.5ex}}}
\newcommand  {\todos}[2][] {\todo{{#1}\begin{itemize}{#2}\end{itemize}}}
\newcommand  {\ignore}[1]  {}

\python
{%@
def fig(name, caption, width=''):
    :\begin{figure}
    :\ffigbox[\FBwidth]
    :{\includegraphics[#(width)]{data/#(name)}}
    :{\caption{\draft{\bf\tiny[#(name)] }#(caption)}\label{fig:#(name)}}
    :\end{figure}
}%@

\python
{%@
def twofig(name1, cap1, name2, cap2, cap_both, cap_width=''):
    :\begin{figure}
    :\ffigbox[#(cap_width)]
    :{
    :\subfloat[\draft{\bf\tiny[#(name1)] }#(cap1)]%
    :{\label{fig:#(name1)}\includegraphics[width=.45\textwidth]{data/#(name1)}}
    :\subfloat[\draft{\bf\tiny[#(name2)] }#(cap2)]%
    :{\label{fig:#(name2)}\includegraphics[width=.45\textwidth]{data/#(name2)}}
    :}
    :{\caption{#(cap_both)}\label{fig:#(name1):#(name2)}}
    :\end{figure}
}%@

\python
{%@
def allfig(name, caption, cap_width=''):
    :\begin{figure}
    :\ffigbox[#(cap_width)]
    :{
    :\subfloat[$\epsilon=10^{-1}$]{\label{fig:#(name):a}\includegraphics[width=.29\textwidth]{data/#(name),EPS-1}}
    :\subfloat[$\epsilon=10^{-4}$]{\label{fig:#(name):b}\includegraphics[width=.29\textwidth]{data/#(name),EPS-4}}
    :\subfloat[$\epsilon=10^{-14}$]{\label{fig:#(name):c}\includegraphics[width=.29\textwidth]{data/#(name),EPS-14}}
    :}
    :{\caption{\draft{\bf\tiny[#(name)] }#(caption)}\label{fig:#(name)}}
    :\end{figure}
}%@

\pgfdeclarepatternformonly{inside domain}%
% bounding box lower left
{\pgfqpoint{-1pt}{-1pt}}%
% bounding box upper right
{\pgfqpoint{10pt}{10pt}}%
% step vector (cell spacing)
{\pgfqpoint{9pt}{9pt}}%
% pattern code
{
  \pgfsetlinewidth{0.4pt}
  \pgfpathmoveto{\pgfqpoint{0pt}{0pt}}
  \pgfpathlineto{\pgfqpoint{9.1pt}{9.1pt}}
  \pgfusepath{stroke}
}

\newlength{\fracheight}
\settoheight{\fracheight}{$\frac{T}{x}$}
\newcommand{\fracspacer}{\rule[-\fracheight]{0pt}{0pt}}

\python
{%@
def version():
    try:
        # For regression testing purposes.
        cmd = 'git describe --all --long %s' % _REV
    except:
        cmd = 'git describe --dirty=+ --all --long'
    ver = shell_eval(cmd).strip()
    if ver[-1] == '+':
        ver = ver[-8:]
    else:
        ver = ver[-7:]
    try:
        # If BASE_REV is set, output in latexdiff format
        base_ver = shell_eval('git rev-parse --short '+BASE_REV).strip()
        return : \today{} $\cdot$ \DIFdel{#(base_ver)} \DIFadd{#(ver)}
    except:
        return : \today{} $\cdot$ #(ver)
}%@

\date{\today}

\newcommand{\ptikz}[1]{\protect\tikz{\protect #1}}

\newcommand{\BJ}{\ensuremath{\R{SJ}}}
\newcommand{\GS}{\ensuremath{\R{GS}}}
\newcommand{\SGS}{\ensuremath{\R{SGS}}}
\newcommand{\cBJ}{\ensuremath{\R{CJ}}}
\newcommand{\cSGS}{\ensuremath{\R{CSGS}}}
\newcommand{\sBJ}{\ensuremath{\R{GJ($-1$)}}}
\newcommand{\sGS}{\ensuremath{\R{GGS}}}
\newcommand{\sSGS}{\ensuremath{\R{GSGS}}}
\newcommand{\GJ}[1]{\ensuremath{\R{GJ($#1$)}}}
\newcommand{\sSGScg}{\ensuremath{\R{GSGS/CG}}}

\newsavebox{\mybox}
\newlength{\myboxlen}
\python
{%@
def capbox(caption, width, adj='-1ex'):
    :\sbox{\mybox}{width}
    :\settowidth{\myboxlen}{\usebox{\mybox}}
    :\addtolength{\myboxlen}{#(adj)}
    :\tabcap{\myboxlen}
    if caption:
        :\caption{#(caption)}
    :\usebox{\mybox}
}%@
